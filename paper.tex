\documentclass[acmsmall,screen]{acmart}

\settopmatter{printacmref=false}

\acmJournal{PACMPL}
\acmVolume{1}
%\acmNumber{POPL}
\acmArticle{1}
%\acmYear{2021}
%\acmMonth{1}
\acmDOI{}
\startPage{1}

\newcommand\hmmax{0} % http://tex.stackexchange.com/questions/3676
\newcommand\bmmax{0}

\usepackage{float}
\usepackage{mathpartir}
\usepackage{bm}
\usepackage{soul}
\usepackage{xspace}
\usepackage{mathtools}
\usepackage{tikz-cd}
\usetikzlibrary{positioning, shapes, arrows, fit}
%\usepackage{breakurl}
\usepackage[scaled=0.75]{beramono}
\usepackage{calrsfs}
\usepackage{subdepth}
\usepackage{dashundergaps}
\usepackage[final]{listings}
\usepackage{mleftright}
\usepackage{extarrows}
\usepackage{placeins}
\usepackage[a]{esvect}
\usepackage{colortbl}
\usepackage{tabularx}
\usepackage{multirow}
\usepackage{subcaption}
\usepackage{enumitem}
\usepackage{longtable}
% \usepackage{sfmath}
\usepackage{pifont}
\usepackage{cancel}
\usepackage{wrapfig}

\newcommand*{\derivationWidth}{0.68\textwidth}
\input{tex-common/proof-helpers}
\input{tex-common/localref}
\input{tex-common/column-types}
\input{tex-common/relational-composition}

\input{tex-common/macros}

\bibliographystyle{ACM-Reference-Format}
\citestyle{acmauthoryear}
\setcitestyle{nosort}

\begin{document}
\title[Automatic Expression Generation For Explainable Science]{Automatic Expression Generation For Explainable Science}
\maketitle

\section{Linguistics}
For a starting point in our investigation into the (hopefully limited) side of linguistics we'll need to explore, I'm 
beginning by taking a look at \citet{poesio23}. It provides what seems like a comprehensive review treatment of anaphors
in computational linguistics. 

A coreference chain is a cluster of mentions, ie pronouns that all refer to the same entity. In our case, we might
expect that an explanation related to a visualization or computation comprises of a number of references to (parts of)
the same output. For example, when providing an explanation of a view, our expressions all refer to the same view, or
set of views. Unsure how to model parts of an object, but intuitively, it feels like a co-reference chain would
refer to the enclosing scope of their constituent parts, ie a \kw{LineChart} constructor is implicitly referred to
whenever we have an expression that relates to a component of the chart.

\bibliography{tex-common/bib}
\end{document}