\section{Introduction}

When reading a scientific article, the reader's comprehension of the text may
be greatly improved by inclusion of charts, tables and figures that help summarise information.
For example, in \figref{table-explanation}, findings from climate change models
are summarized in a table with an accompanying paragraph providing the reader
with context for interpeting the table, references to ``SSP-'' in the text meaning
the rows of the table, with column of interest left implicit at first.

Visualizations such as tables and bar-charts summarize data to make it easier to
understand. The readers comprehension can be further aided by incorporating 
interactivity into the visualization. By enabling the reader to click on parts
of a chart to see the data it summarizes, the reader gets a better sense of what
the author is trying to say. Understanding can be improved further, by extending
interactivity to the text itself.

Authoring of such articles is difficult and time-consuming, and taking an article
and building interactive features into its visualisations requires a further investment
of effort, often on an ad-hoc basis. As yet, there has been no tool that extends
interactivity features to treat text as an interactive visual element. It is not
entirely clear what such features look like when extended to text, and requiring
the author to spend even more time effectively programming their article imposes
a significant burden on the author. 

Some of the difficulty can be mitigated by building visualizations in a language
which provides automatic support for interaction, but these do not provide tools
or support for linking data directly to the text of an article. An article will
be full of references to items of discourse, and asking the author to manually
link each reference to its referent for the purposes of interaction is liable to
error, or the user simply giving up. An automated tool is required. 

\begin{figure}
   \includegraphics[width=0.9\textwidth]{fig/ipcc-table-explanation.png}
   \caption{Explanation based on the contents of a table}
   \label{fig:table-explanation}
\end{figure}